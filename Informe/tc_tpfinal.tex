\documentclass[../tc_tpfinal_main.tex]{subfiles}

\begin{document}

%capítulo
\chapter{Filtro}

\section{Introducción} 

\section{Análisis de sensibilidades}
\subsection{Celda Sallen-Key Pasabandas}

\begin{figure}[H]
	\centering
	
\begin{circuitikz}
  	\draw (0,0) node[op amp, yscale=-1] (opamp) {}
  		(opamp.-) -| (-1.5, -1.5) node[left] {$V^-$}
  					to [R = $R_3$, *-]  (-1.5, -3.5) node[ground] {}
  		
  		(-1.5,-1.5) to [R = $R_4$] (1.5, -1.5) 
  					to [short, -*] (1.5, 0) 
  					to [short, -o] (2, 0) node[right] {$V_{out}$}
  		(1.5,0) 	to [short] (opamp.out) 
  		
  		(opamp.+) 	to [short, -*] (-2.5, 0.5) node[above]{$V^+$}
  					to [R, l=$R_2$] (-2.5, -1.5) node[ground]{}
  		
		(-2.5, 0.5) to [C, l_=$C_2$] (-4.5, 0.5)
					to [R, l=$R_1$, *-o] (-6.5, 0.5) node[left]{$V_{in}$}  		
		(-4.5, 0.5) to [C, l=$C_1$] (-4.5, -1.5) node[ground]{}
	
		(-4.5, 0.5) to [short, -*] (-4.5, 2) node[above] {$V_f$}
					to [R, l=$R_f$] (1.5,2)
					to [short] (1.5,0)
  		;
\end{circuitikz}
	\caption{Celda Sallen-Key pasabanda}
	\label{fig:sk_lp_schem}
\end{figure}

\begin{table}[H] 
	\centering
 				\begin{tabular}{||c c c c c c c c c||} 
 					\hline
	  Parámetro& $R_1$ & $R_2$ & $R_3$ & $R_4$ & $r_a$ & $r_b$&$C_1$&$C_2$\\ [0.5ex] 
 					\hline\hline
		 $S^G_x$& 1 & 0& -1& 0&0&0&0&0\\
		 $S^{w_0}_x$& $- \frac{1}{2}$ &$- \frac{1}{2}$& 0& 0&$- \frac{1}{2}$&$- \frac{1}{2}$&$- \frac{1}{2}$&$- \frac{1}{2}$\\
		 $S^{Q}_x$&$- \frac{1}{2}$ &$- \frac{1}{2}$& 0& 1&$- \frac{1}{2}$&$- \frac{1}{2}$&$- \frac{1}{2}$&$- \frac{1}{2}$\\[1ex] 
		\hline
	\end{tabular}
\end{table}
\subsection{Celda Sallen-Key Pasa-altos}
\begin{figure}[H]
	\centering
	
\begin{circuitikz}
  	\draw (0,0) node[op amp, yscale=-1] (opamp) {}
  		(opamp.-) -| (-1.5, -1.5) 
  		 			to [short] (1.5, -1.5) 
  					to [short, -*] (1.5, 0) 
  					to [short, -o] (2, 0) node[right] {$V_{out}$}
  		(1.5,0) 	to [short] (opamp.out) 
  		
  		(opamp.+) 	to [short, -*] (-2.5, 0.5) node[above]{$V^+$}
  					to [R, l=$R_2$] (-2.5, -1.5) node[ground]{}
  		
		(-2.5, 0.5) to [C, l=$C_2$] (-4.5, 0.5)
					to [C, l=$C_1$, *-o] (-6.5, 0.5) node[left]{$V_{in}$}  		

	
		(-4.5, 0.5) to [short, -*] (-4.5, 2) node[above] {$V_f$}
					to [R, l=$R_f$] (1.5,2)
					to [short] (1.5,0)
  		;
\end{circuitikz}
	\caption{Celda Sallen-Key pasaaltos}
	\label{fig:sk_hp_schem}
\end{figure}



\begin{table}[H] 
	\centering
 	\begin{tabular}{||c c c c c c c c c||} 
 		\hline
	 	Parámetro& $R_1$ & $R_2$ & $R_3$ & $R_4$ & $r_a$ & $r_b$&$C_1$&$C_2$\\ [0.5ex] 
 		\hline\hline
		 $S^G_x$& 1 & 0& -1& 0&0&0&0&0\\
		 $S^{w_0}_x$& $- \frac{1}{2}$ &$- \frac{1}{2}$& 0& 0&$- \frac{1}{2}$&$- \frac{1}{2}$&$- \frac{1}{2}$&$- \frac{1}{2}$\\
		 $S^{Q}_x$&$- \frac{1}{2}$ &$- \frac{1}{2}$& 0& 1&$- \frac{1}{2}$&$- \frac{1}{2}$&$- \frac{1}{2}$&$- \frac{1}{2}$\\[1ex] 
		\hline
	\end{tabular}
\end{table}


\subsection{Celda Sallen-Key Pasa-altos con factor ganancia}


\begin{figure}[H]
	\centering
	
\begin{circuitikz}
  	\draw (0,0) node[op amp, yscale=-1] (opamp) {}
  		(opamp.-) -| (-1.5, -1.5) 
  					to [R = $R_3$, *-]  (-1.5, -3.5) node[ground] {}
  		(-1.5,-1.5) to [R = $R_4$] (1.5, -1.5)   		 			
  		 			 
  					to [short, -*] (1.5, 0) 
  					to [short, -o] (2, 0) node[right] {$V_{out}$}
  		(1.5,0) 	to [short] (opamp.out) 
  		
  		(opamp.+) 	to [short, -*] (-2.5, 0.5) node[above]{$V^+$}
  					to [R, l=$R_2$] (-2.5, -1.5) node[ground]{}
  		
		(-2.5, 0.5) to [C, l=$C_2$] (-4.5, 0.5)
					to [C, l=$C_1$, *-o] (-6.5, 0.5) node[left]{$V_{in}$}  		

	
		(-4.5, 0.5) to [short, -*] (-4.5, 2) node[above] {$V_f$}
					to [R, l=$R_f$] (1.5,2)
					to [short] (1.5,0)
  		;
\end{circuitikz}

	\caption{Celda Sallen-Key pasaaltos con factor de ganancia}
	\label{fig:sk_hp_gain_schem}
\end{figure}
\begin{table}[H] %datos thd simulado
	\centering
	\begin{tabular}{||c c c c c c c c c||} 
 		\hline
	  	Parámetro& $R_1$ & $R_2$ & $R_3$ & $R_4$ & $r_a$ & $r_b$&$C_1$&$C_2$\\ [0.5ex] 
 		\hline\hline
		$S^G_x$& 1 & 0& -1& 0&0&0&0&0\\
		$S^{w_0}_x$& $- \frac{1}{2}$ &$- \frac{1}{2}$& 0& 0&$- \frac{1}{2}$&$- \frac{1}{2}$&$- \frac{1}{2}$&$- \frac{1}{2}$\\
		$S^{Q}_x$&$- \frac{1}{2}$ &$- \frac{1}{2}$& 0& 1&$- \frac{1}{2}$&$- \frac{1}{2}$&$- \frac{1}{2}$&$- \frac{1}{2}$\\[1ex] 
		\hline
	\end{tabular}
\end{table}
\subsection{Celda Tow-Thomas}

\begin{figure}[H]	
	\centering
	
\begin{circuitikz}
  	\draw (0,0) 
  	
%primer opamp
  		node[op amp, yscale=1] (oa1) {}
  		(oa1.+) -| (-1.5, -1.5) node[ground]{}
  		(1.5,0) 	to [short] (oa1.out) 
  		
  		(oa1.-) 	to [short,-*] (-1.5,0.5)
  					to [R, l=$R_3$] (-3.5, 0.5) 
  					to [short,-o] (-4, 0.5) node[left]{$V_{in}$}
%feedback primer opamp
		(-1.5,0.5) 	to [short] (-1.5,2)
					to [R,l_=$R_4$] (1.5,2)
					to [short] (1.5,0)
					
		(-1.5,2)	to [short] (-1.5,3)
					to [C,l=$C_1$] (1.5,3)
					to [short] (1.5,2)
						
%segundo opamp
		(5,0) node[op amp, yscale=1] (oa2) {}
		(oa2.-) 	to [R, l=$R_4$] (1.5,0.5)
		(oa2.+) -| (3.5, -1.5) node[ground]{}

%feedback segundo opamp
		(3.5,0.5)	to [short] (3.5,2)
					to [C, l=$C_2$] (6.5,2)
					|- (oa2.out)
					
					
%tercer opamp
		(10,0) node[op amp, yscale=1] (oa3) {}
		(oa3.-) 	to [R, l=$R_5$] (6.5,0.5)
		(oa3.+) -| (8.5, -1.5) node[ground]{}

%feedback tercer opamp
		(8.5,0.5)	to [short] (8.5,2)
					to [R, l=$R_6$] (11.5,2)
					|- (oa3.out)
		
%feedback global
		(11.5,2)	to [short] (11.5,4)
					to [R,l=$R_2$] (-1.5,4)
					to [short] (-1.5,2)
%salida
		(11.5,0)	to [short, -o] (12,0) node[right] {$V_{out}$}				
		
  		;
\end{circuitikz}
	\caption{Celda Tow-Thomas}
	\label{fig:tpfinal_tow_thomas_circ}
\end{figure}

Se despeja la transferencia total del sistema:\par
\begin{center}
$H(s) = -\frac{\mathrm{R_1}\, \mathrm{R_4}\, \mathrm{r_b}}{\mathrm{R_3}\, \left(\mathrm{C_1}\, \mathrm{C_2}\, \mathrm{R_1}\, \mathrm{R_2}\, \mathrm{R_4}\, \mathrm{r_b}\, s^2 + \mathrm{C_2}\, \mathrm{R_1}\, \mathrm{R_2}\, \mathrm{r_a}\, s + \mathrm{R_4}\, \mathrm{r_b}\right)}
$
\end{center}

De la cual se despejan los siguientes parámetros:\par

\begin{center}
$w_0 = \sqrt{\frac{r_b}{C_1\cdot C_2\cdot R_1\cdot R_2\cdot ra}}; $
%wo = sqrt(rb/(c1*c2*r1*r2*ra))
$Q = \sqrt{\frac{C_1\cdot r_b}{C_2\cdot R_1\cdot R_2\cdot ra}}; $
%Q = r4*sqrt(c1*rb/(c2*r1*r2*ra));
$G = -\frac{R_1}{R_4}; $ 
\end{center}




Para la ganancia, obtenemos las sensibilidades con respecto a todos los componentes:\par

\begin{table}[H] %datos thd simulado
	\centering
 	\begin{tabular}{||c c c c c c c c c||} 
 		\hline
	  	Parámetro& $R_1$ & $R_2$ & $R_3$ & $R_4$ & $r_a$ & $r_b$&$C_1$&$C_2$\\ [0.5ex] 
 		\hline\hline
		$S^G_x$& 1 & 0& -1& 0&0&0&0&0\\
		$S^{w_0}_x$& $- \frac{1}{2}$ &$- \frac{1}{2}$& 0& 0&$- \frac{1}{2}$&$- \frac{1}{2}$&$- \frac{1}{2}$&$- \frac{1}{2}$\\
		$S^{Q}_x$&$- \frac{1}{2}$ &$- \frac{1}{2}$& 0& 1&$- \frac{1}{2}$&$- \frac{1}{2}$&$- \frac{1}{2}$&$- \frac{1}{2}$\\[1ex] 
		\hline
	\end{tabular}
\end{table}
			


\end{document}
